The core of this work is the  numerical simulation of a quantum
harmonic oscillator on the lattice, with the purpose of recovering the analytical results both for the discretized theory and in the continuum.
Namely, we will compute the observables
\begin{align}
  \label{eqn:phys1}
  \DEt \equiv \Et{1}-\Et{0}=\wt\\
  \matb = \frac{1}{2m\wb} \label{eqn:phys2}
\end{align}
and use them to extrapolate the equivalent quantities of the continuum theory sending $a\to 0$.

In this limit, the lattice results differ from the continuum quantities by linear corrections in $a ^2$

\begin{align}
  &\DEt \approx \w - \frac{\w ^{3}}{24}a ^2 \label{eqn:fullen}\\
  &\matb \approx \frac{1}{2m\w} - \frac{\w}{16m}a ^2 \label{eqn:fullmat}
\end{align}
so we need to choose a small enough lattice spacing to approximate the continuum limit, while keeping
the product $Na$ sufficiently large that the ground state properties of the system can still be observed.
If we define $T_{E}\equiv \nicefrac{2\pi}{E_{0}}$, a sensible choice according to \cite{Creutz1981ASA} is to consider $a\sim 0.1,0.2 \times T_{E}$ and $Na \sim 3,10 \times T_{E}$. In this work we kept $Na=64$ and we performed the same simulation on $5$ different lattices with progressively smaller spacing $a$, eventually obtaining the continuum ($a=0$) results after a linear fit on $a ^2$.

These quantum properties can be extracted from the two points correlation function of the position operator:
if we select two sites $l$ and $k$ on the lattice, such that $\t \equiv a\abs{l-k}$\footnote{Physical time $t$ will always appear in unit of $a$.},
we can define
\begin{align}
\label{eqn:integral}
\ct \equiv  \ev{x_{l}x_{k}} = \frac{1}{Z(T,a)}\int \prod_{i=0}^{N-1}\dd{x}_{i}x_{l}x_{k}e ^{-S}
\end{align}
as our primary observables, which we intend to evaluate with numerical integration methods.
The integral is invariant under translation, therefore the correlator is independent of the
particular sites $l$, $k$ and actually depends on their distance $\lk = \nicefrac{\t}{a}$.

We can rewrite the correlator with the operator formalism
\begin{align}
  \ct = \frac{1}{\Tr[\T{N}]}\Tr[\T{N-\lk}\x\T{\lk}\x];
\end{align}
and, after inserting two complete sets of $\T{}$ eigenstates, the dominant contribution
in the large $\lk$ and $N$ limit is found to be
\begin{align}
  \label{eqn:correlators}
  \ct \approx& 2\matb \exp\qty(-\frac{N}{2}a\DEt)\times\notag\\&\times\cosh\qty[\qty(\frac{N}{2}-|l-k|)a\DEt].
\end{align}
This expression is even with respect to $|l-k| = \nicefrac{N}{2}$, so we can limit our analysis to the correlators whose distance  $|l-k|$ is at most $\nicefrac{N}{2}$.\\
To find the energy gap without having to fit the data using equation \ref{eqn:correlators} as the model, we consider a combination
of correlators at different physical times and invert it to find the analytic expression
\begin{align}
  &\DEt = \frac{1}{a}\arccosh\qty[\frac{\ctp + \ctm}{2\ct}] \label{eqn:energy}
\end{align}
which can be plugged back into equation \ref{eqn:correlators} to find the matrix element
\begin{align}
  &\matb = \frac{\ct\exp(\frac{N}{2}a\DEt)}{2\cosh\qty[\qty(\frac{N}{2}-|l-k|)a\DEt]}\label{eqn:matelem}.
\end{align}

Moreover, we can also prove that the noise to signal ratio of the correlator
grows exponentially with $\t$  and therefore only small values of $\t$ are going to be useful for
our analysis. This was highlighted by Giorgio Parisi in $1983$ and was given the name of \textit{exponential problem}.

Assuming the observables $\ct$ are averaged over $N_{config}$ configurations, the variance is
\begin{align}
  \label{eqn:err1}
  \sigma ^2\qty[\ct]= \frac{1}{N_{config}}\qty[\ev{x_{l}^2x_{k}^2}-\ev{x_{l}x_{k}}^2],
\end{align}
where $\ev{x_{l}x_{k}}$ is precisely $\ct$, while $\ev{x_{l}^2x_{k}^2}$ is the two points correlation function of the
$\x ^2$ operator. \\Like we did for the correlator of $\hat{x}$, we can write $\ev{x_{l}^{2}x_{k}^{2}}$ using
the operator formalism and, following the same steps, recover its dominant contribution for large $N$
\begin{align}
  \ev{x_{l}^{2}x_{k}^{2}} = &\frac{1}{\Tr[\T{N}]}\Tr[\T{N-\lk}\x^{2}\T{\lk}\x^{2}] \notag\\
  &\approx \abs{\mel{\Et{0}}{\x ^2}{\Et{0}}}^2\label{eqn:xlxkq}.
\end{align}
Before we continue discussing the exponential problem, it is worth noting that for large values of $N$, as the distance $|l-k|$ grows, the variance (equation \ref{eqn:err1})
will be dominated by the constant value of $\ev{x_{l}^{2}x_{k}^{2}}$\footnote{Equation \ref{eqn:correlators} shows that $\ct$ is suppressed exponentially with $N$ when $|l-k|$ is large enough.}
\begin{align}
  \label{eqn:varmatrix}
  \sigma^{2}[\ct]\approx \frac{ \abs{\mel{\Et{0}}{\x ^2}{\Et{0}}}^2}{N_{config}},
\end{align}
meaning that it would be possible to use the variance of our numerical computation to extract the ground state expectation value of $\x^{2}$.\\
On the other hand, if we focus on the region where $N$ and $\lk$ are large but $\lk \ll \nicefrac{N}{2}$, equation \ref{eqn:correlators} implies that the correlators decrease very rapidly with $\lk$
\footnote{$\cosh\qty(x)\sim \nicefrac{e^{x}}{2}$ as $x\to\infty$.}
\begin{align}
  \label{eqn:errct}
  \ct \approx \matb e ^{-a\lk \DEt}.
\end{align}
The noise to signal ratio, \textit{i.e.} the relative error, therefore grows exponentially with the distance of the sources
\begin{align}
  \label{eqn:relerr}
 \frac{\sigma ^2\qty[\ct]}{\ct ^2}\approx \frac{1}{N_{config}}\frac{\abs{\mel{\Et{0}}{\x ^2}{\Et{0}}}^2}{\abs{\mel{\Et{0}}{\x}{\Et{1}}}^4}e ^{2a\lk\DEt}.
\end{align}

This behaviour greatly limits the number of correlators we can actually use to calculate the energy gaps and matrix elements (equations \ref{eqn:energy} and \ref{eqn:matelem}) for a given number of configurations, as the value of the observables is eventually going to be dominated by their error.
A possible solution would be to exponentially increase the number of configurations, which would heavily worsen the computational cost.

More generally, the origin of this problem in QFTs is to be found in the quantization of the energy spectrum which leads to the presence of a mass gap in the theory.
