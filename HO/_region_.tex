\message{ !name(intro.tex)}
\message{ !name(intro.tex) !offset(-2) }
\section{Introduction}%
\label{sec:Introduction}
The harmonic oscillator is likely the simplest physical system that is also worth studying. Despite said simplicity, a deep understanding of the harmonic oscillator is the key to understand complex topics such  as classical mechanics and quantum field theory.

Its numerical resolution is particularly relevant in computational physics, as it provides the blueprint for the study of quantum field theories on the lattice with the path integral formalism.


\subsection{One dimensional quantum harmonic oscillator}%
\label{subsec:1DQHO}
The hamiltonian of a one dimensional quantum harmonic oscillator of mass $m$ and frequency $\nicefrac{\omega}{2\pi}$ is
\begin{align}
  \label{eqn:1}
  \hat{H}= \frac{\hat{p}^{2}}{2m} + \frac{1}{2}m\omega  ^2 \hat{x}
\end{align}
with eigenvalues $E_{n}$, eigenstates $\ket{E_{n}}$ and position operator eigenfunctions $u_{n}(x)$
\begin{align}
  \label{eqn:2}
  &E_{n}=\w \left(n+\frac{1}{2}\right) \\
  &u_{n}(x) = \braket{\hat{x}}{E_{n}} = \notag \\
    = \left(\frac{m\w}{\pi}\right)^{\frac{1}{4}}\left(\frac{1}{n}\right)^{\frac{1}{2}}
              e ^{-\frac{1}{2}m\w x ^2} He_{n}(x \sqrt{2m\w})
\end{align}
where $He_{n}(x)$ is the Hermite polynomial of degree $n$.

Exploting the property under parity transformations of Hermite polynomials \footnote{$He_{n}(-x) = - He_{n}(x)$.}



\subsection{Purpose of the simulation}%
\label{subsec:purpose}

\message{ !name(intro.tex) !offset(-34) }
