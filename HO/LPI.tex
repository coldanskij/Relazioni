Feynman's formulation of quantum mechanics reveals a connection
between the quantum theory and statistical mechanics through the equivalence of the
Euclidean path integral and the partition function of a statistical system with Boltzmann weight $e^{-S}$, where $S$ is the classical action
we intend to quantize.
As a result, we are able to apply the well known
Monte Carlo techniques of statistical physics to quantum mechanics\footnote{And potentially to quantum field theories.},
provided we can rigorously define the path integral.

The only way to do so is to perform a discretization of Euclidean spacetime with cells of size $a$, put it in a box of finite size $Na$ and
work at finite values of $a$ and $N$, then -- at a later stage -- take the limits $a\to 0$, $N\to\infty$ of the results\footnote{Eventually we can \textit{rotate} back to the Minkowski spacetime.}.

\begin{figure}[h]
\centering
\ctikzfig{timedi}
\caption{Discretization of the time interval -- we have defined a one dimensional lattice of $N$ sites and spacing $a$ with periodic boundary conditions.}
\label{fig:Timediscr}
\end{figure}

For one dimensional systems, we are left with a lattice with $N$ discrete time coordinates $t_{i}$, with $i = 0,\dots, N-1$. Therefore we define the position coordinate at time $t_{i}$ to be $x_{i}\equiv x(t_{i})$,
from which we define the discretized versions of the harmonic oscillator Langrangian and the corresponding action
\begin{align}
  &\Lag = \frac{m}{2}\left(\frac{x_{i+1}-x_{i}}{a}\right)^2 + \frac{1}{4}m\w ^2 \left(x_{i+1}^2+x_{i}^2\right)\label{eqn:dlag}\\
  &S = a\sum_{i=0}^{N-1}\Lag.
\end{align}
The lattice partition function of such a system is
\begin{align}
  \label{eqn:Zlattice}
  Z(T,a) &\equiv \left(\frac{m}{2\pi a}\right)^{\frac{N}{2}}\int \prod_{i=0}^{N-1} \dd{x}_{i}e ^{-S} \notag\\
  &= \Tr[\T{N}] ,
\end{align}
where $\T{}$ is the transfer operator
\begin{align}
  &\T{} = \exp\qty[-\frac{a}{2}V(\x)]\exp\qty(-a \frac{\p ^2}{2m})\exp\qty[-\frac{a}{2}V(\x)], \notag \\
   &V(\x) = \frac{1}{2}m\w^{2}\x^{2}
\end{align}
which evolves the system on the lattice of one step of size $a$. This is the analogous to the time evolution operator of the continuum theory and,
as it is explained in \cite{Creutz1981ASA}, its
diagonalization is equivalent to finding the time evolution of the discretized system.

The transfer operator is unitary by construction, therefore it is possible to express it as the exponentiation of a self adjoint operator with eigenvalues $\varepsilon_{n}$ and eigenstates
$\ket{\varepsilon_{n}}$
\begin{align}
  \label{eqn:Teig}
  \T{} \ket{\varepsilon_{n}} = e ^{-a\varepsilon_{n}}\ket{\varepsilon_{n}}.
\end{align}
To diagonalize it we introduce a hamiltonian $\Hb$
\begin{align}
  &\Hb = \frac{\p ^2}{2m}+\frac{1}{2}m\wb ^2 \x ^2,\\
  &\wb ^2 \equiv \w ^2 \left(1 + \frac{a ^2 \w ^2}{4}\right)
\end{align}
where $\w$ is the same frequency that appears in the Lagrangian \ref{eqn:dlag}.
This is an auxiliary harmonic oscillator on the continuum, whose frequency $\wb$ depends on $a$, with energy levels
\begin{align}
  &\Et{n} = \wt \left(n + \frac{1}{2}\right), \\
  &\wt   = \frac{1}{a}\log \left(1 + a\wb + \frac{a ^2\w ^2}{2}\right) \label{eqn:expen}
\end{align}
which, in the limit $a\to 0$, rebuild the spectrum of the \textit{original} harmonic oscillator.

Since $\Hb$ and $\T{}$ commute, they share a basis of eigenstates $\qty{\ket{\Et{n}}}$ with eigenvalues that are respectively $\Et{n}$ and $e^{-a\Et{n}}$. Therefore
combining equation \ref{eqn:Teig} and the definition \ref{eqn:Zlattice} we explicitely find the partition function both on the lattice and in the continuum
\begin{align}
  Z(T,a)= \sum_{n=0}^{\infty}e ^{-a\Et{n}N} \underbrace{\longrightarrow}_{\underset{a\to 0}{N\to \infty}} Z(T) = \sum_{n=0}^{\infty}e ^{-E_{n}T}.
\end{align}
Finally, simply by substituting $\wb$ to $\w$ in equations \ref{eqn:mat} we find the useful matrix elements
\begin{align}
  &\ev{\hat{x}}{\Et{0}}=0,\quad
  \mel{\Et{0}}{\hat{x}}{\Et{1}} = \frac{1}{\sqrt{2m\wb}}, \notag\\
  &\ev{\hat{x}^2}{\Et{0}}= \frac{1}{2m\wb} \label{eqn:matbar}.
\end{align}
