\section{Introduction}%
\label{sec:Introduction}
The harmonic oscillator is likely the simplest physical system that is also worth studying. Despite said simplicity, a deep understanding of the harmonic oscillator is the key to understand complex topics such  as classical mechanics and quantum field theory.

The path integral approach is particularly relevant in computational physics, as it provides the\\ blueprint for the study of quantum field theories on the lattice.
The steps taken to compute the energy gap between the ground and the first excited state of the harmonic oscillator are the same ones that are
required to compute the mass of the lightest particle with a particular set of quantum numbers of a given theory -- \textit{e.g.}
the proton in QCD -- and the results present the great advantage of being able to be confronted with known analytical values, both
on the lattice and in the continuum.
This allows us to verify that the lattice numerical resolution, which we implemented with Monte Carlo techniques, is correct in both cases.

\subsection{One dimensional quantum harmonic oscillator}%
\label{subsec:1DQHO}
The hamiltonian of a one dimensional quantum harmonic oscillator of mass $m$ and frequency $\nicefrac{\omega}{2\pi}$ is
\begin{align}
  \label{eqn:1}
  \H= \frac{\hat{p}^{2}}{2m} + \frac{1}{2}m\omega  ^2 \hat{x}^2
\end{align}
with the position and momentum operators obeying the canonical commutation relations $\comm{\x}{\p}=i$\footnote{In \textit{natural units} where $\hbar = c = K_{B} = 1$.}
and its eigenvalues $E_{n}$ and eigenstates $\ket{E_{n}}$ defined by the equation
\begin{align}
  \label{eqn:diago}
  \H \ket{E_{n}} = E_{n}\ket{E_{n}}.
\end{align}
It is convenient to introduce the ladder operators $\ad$ and $\a$
\begin{align}
  &\ad \equiv \sqrt{\frac{m\w}{2}}\left(\hat{x} - i\frac{\p}{m\w}\right),\\
  &\a \equiv \sqrt{\frac{m\w}{2}}\left(\x + i \frac{\p}{m\w}\right),\\
  &\comm{\a}{\ad}=1,\\
  &\a \ket{E_{n}} = \sqrt{n}\ket{E_{n-1}},\\
  &\ad  \ket{E_{n}} = \sqrt{n+1}\ket{E_{n+1}}
\end{align}
which can be used to rewrite $\H$ and easily find its eigenvalues
\begin{align}
  &\H = \w \left(\ad \a + \frac{1}{2}\right), \
  E_{n} = \w \left(n+\frac{1}{2}\right).
\end{align}
Rewriting $\x$ in terms of $\a$ and $\ad$, the computation of matrix elements is also straightforward,
in particular we will make use of
\begin{align}
  &\ev{\hat{x}}{E_{0}}=0,\quad
  \mel{E_{0}}{\hat{x}}{E_{1}} = \frac{1}{\sqrt{2m\w}}, \notag\\
  &\ev{\hat{x}^2}{E_{0}}= \frac{1}{2m\w} \label{eqn:mat}.
\end{align}


\subsection{Lattice path integral for the harmonic oscillator}%
\label{sec:LPI}
Feynman's formulation of quantum mechanics reveals a connection
between the quantum theory and statistical mechanics through the equivalence of the
Euclidean path integral and the partition function of a statistical system with Boltzmann weight $e^{-S}$, where $S$ is the classical action
we intend to quantize.
As a result, we are able to apply the well known
Monte Carlo techniques of statistical physics to quantum mechanics\footnote{And potentially to quantum field theories.},
provided we can rigorously define the path integral.

The only way to do so is to perform a discretization of Euclidean spacetime with cells of size $a$, put it in a box of finite size $Na$ and
work at finite values of $a$ and $N$, then -- at a later stage -- take the limits $a\to 0$, $N\to\infty$ of the results\footnote{Eventually we can \textit{rotate} back to the Minkowski spacetime.}.

\begin{figure}[h]
\centering
\ctikzfig{timedi}
\caption{Discretization of the time interval -- we have defined a one dimensional lattice of $N$ sites and spacing $a$ with periodic boundary conditions.}
\label{fig:Timediscr}
\end{figure}

For one dimensional systems, we are left with a lattice with $N$ discrete time coordinates $t_{i}$, with $i = 0,\dots, N-1$. Therefore we define the position coordinate at time $t_{i}$ to be $x_{i}\equiv x(t_{i})$,
from which we define the discretized versions of the harmonic oscillator Langrangian and the corresponding action
\begin{align}
  &\Lag = \frac{m}{2}\left(\frac{x_{i+1}-x_{i}}{a}\right)^2 + \frac{1}{4}m\w ^2 \left(x_{i+1}^2+x_{i}^2\right)\label{eqn:dlag}\\
  &S = a\sum_{i=0}^{N-1}\Lag.
\end{align}
The lattice partition function of such a system is
\begin{align}
  \label{eqn:Zlattice}
  Z(T,a) &\equiv \left(\frac{m}{2\pi a}\right)^{\frac{N}{2}}\int \prod_{i=0}^{N-1} \dd{x}_{i}e ^{-S} \notag\\
  &= \Tr[\T{N}] ,
\end{align}
where $\T{}$ is the transfer operator
\begin{align}
  &\T{} = \exp\qty[-\frac{a}{2}V(\x)]\exp\qty(-a \frac{\p ^2}{2m})\exp\qty[-\frac{a}{2}V(\x)], \notag \\
   &V(\x) = \frac{1}{2}m\w^{2}\x^{2}
\end{align}
which evolves the system on the lattice of one step of size $a$. This is the analogous to the time evolution operator of the continuum theory and,
as it is explained in \cite{Creutz1981ASA}, its
diagonalization is equivalent to finding the time evolution of the discretized system.

The transfer operator is unitary by construction, therefore it is possible to express it as the exponentiation of a self adjoint operator with eigenvalues $\varepsilon_{n}$ and eigenstates
$\ket{\varepsilon_{n}}$
\begin{align}
  \label{eqn:Teig}
  \T{} \ket{\varepsilon_{n}} = e ^{-a\varepsilon_{n}}\ket{\varepsilon_{n}}.
\end{align}
To diagonalize it we introduce a hamiltonian $\Hb$
\begin{align}
  &\Hb = \frac{\p ^2}{2m}+\frac{1}{2}m\wb ^2 \x ^2,\\
  &\wb ^2 \equiv \w ^2 \left(1 + \frac{a ^2 \w ^2}{4}\right)
\end{align}
where $\w$ is the same frequency that appears in the Lagrangian \ref{eqn:dlag}.
This is an auxiliary harmonic oscillator on the continuum, whose frequency $\wb$ depends on $a$, with energy levels
\begin{align}
  &\Et{n} = \wt \left(n + \frac{1}{2}\right), \\
  &\wt   = \frac{1}{a}\log \left(1 + a\wb + \frac{a ^2\w ^2}{2}\right) \label{eqn:expen}
\end{align}
which, in the limit $a\to 0$, rebuild the spectrum of the \textit{original} harmonic oscillator.

Since $\Hb$ and $\T{}$ commute, they share a basis of eigenstates $\qty{\ket{\Et{n}}}$ with eigenvalues that are respectively $\Et{n}$ and $e^{-a\Et{n}}$. Therefore
combining equation \ref{eqn:Teig} and the definition \ref{eqn:Zlattice} we explicitely find the partition function both on the lattice and in the continuum
\begin{align}
  Z(T,a)= \sum_{n=0}^{\infty}e ^{-a\Et{n}N} \underbrace{\longrightarrow}_{\underset{a\to 0}{N\to \infty}} Z(T) = \sum_{n=0}^{\infty}e ^{-E_{n}T}.
\end{align}
Finally, simply by substituting $\wb$ to $\w$ in equations \ref{eqn:mat} we find the useful matrix elements
\begin{align}
  &\ev{\hat{x}}{\Et{0}}=0,\quad
  \mel{\Et{0}}{\hat{x}}{\Et{1}} = \frac{1}{\sqrt{2m\wb}}, \notag\\
  &\ev{\hat{x}^2}{\Et{0}}= \frac{1}{2m\wb} \label{eqn:matbar}.
\end{align}


\subsection{Purpose of the simulation}%
\label{subsec:purpose}
The core of this work is the  numerical simulation of a quantum
harmonic oscillator on the lattice, with the purpose of recovering the analytical results both for the discretized theory and in the continuum.
Namely, we will compute the observables
\begin{align}
  \label{eqn:phys1}
  \DEt \equiv \Et{1}-\Et{0}=\wt\\
  \matb = \frac{1}{2m\wb} \label{eqn:phys2}
\end{align}
and use them to extrapolate the equivalent quantities of the continuum theory sending $a\to 0$.

In this limit, the lattice results differ from the continuum quantities by linear corrections in $a ^2$

\begin{align}
  &\DEt \approx \w - \frac{\w ^{3}}{24}a ^2 \label{eqn:fullen}\\
  &\matb \approx \frac{1}{2m\w} - \frac{\w}{16m}a ^2 \label{eqn:fullmat}
\end{align}
so we need to choose a small enough lattice spacing to approximate the continuum limit, while keeping
the product $Na$ sufficiently large that the ground state properties of the system can still be observed.
If we define $T_{E}\equiv \nicefrac{2\pi}{E_{0}}$, a sensible choice according to \cite{Creutz1981ASA} is to consider $a\sim 0.1,0.2 \times T_{E}$ and $Na \sim 3,10 \times T_{E}$. In this work we kept $Na=64$ and we performed the same simulation on $5$ different lattices with progressively smaller spacing $a$, eventually obtaining the continuum ($a=0$) results after a linear fit on $a ^2$.

These quantum properties can be extracted from the two points correlation function of the position operator:
if we select two sites $l$ and $k$ on the lattice, such that $\t \equiv a\abs{l-k}$\footnote{Physical time $t$ will always appear in unit of $a$.},
we can define
\begin{align}
\label{eqn:integral}
\ct \equiv  \ev{x_{l}x_{k}} = \frac{1}{Z(T,a)}\int \prod_{i=0}^{N-1}\dd{x}_{i}x_{l}x_{k}e ^{-S}
\end{align}
as our primary observables, which we intend to evaluate with numerical integration methods.
The integral is invariant under translation, therefore the correlator is independent of the
particular sites $l$, $k$ and actually depends on their distance $\lk = \nicefrac{\t}{a}$.

We can rewrite the correlator with the operator formalism
\begin{align}
  \ct = \frac{1}{\Tr[\T{N}]}\Tr[\T{N-\lk}\x\T{\lk}\x];
\end{align}
and, after inserting two complete sets of $\T{}$ eigenstates, the dominant contribution
in the large $\lk$ and $N$ limit is found to be
\begin{align}
  \label{eqn:correlators}
  \ct \approx& 2\matb \exp\qty(-\frac{N}{2}a\DEt)\times\notag\\&\times\cosh\qty[\qty(\frac{N}{2}-|l-k|)a\DEt].
\end{align}
This expression is even with respect to $|l-k| = \nicefrac{N}{2}$, so we can limit our analysis to the correlators whose distance  $|l-k|$ is at most $\nicefrac{N}{2}$.\\
To find the energy gap without having to fit the data using equation \ref{eqn:correlators} as the model, we consider a combination
of correlators at different physical times and invert it to find the analytic expression
\begin{align}
  &\DEt = \frac{1}{a}\arccosh\qty[\frac{\ctp + \ctm}{2\ct}] \label{eqn:energy}
\end{align}
which can be plugged back into equation \ref{eqn:correlators} to find the matrix element
\begin{align}
  &\matb = \frac{\ct\exp(\frac{N}{2}a\DEt)}{2\cosh\qty[\qty(\frac{N}{2}-|l-k|)a\DEt]}\label{eqn:matelem}.
\end{align}

Moreover, we can also prove that the noise to signal ratio of the correlator
grows exponentially with $\t$  and therefore only small values of $\t$ are going to be useful for
our analysis. This was highlighted by Giorgio Parisi in $1983$ and was given the name of \textit{exponential problem}.

Assuming the observables $\ct$ are averaged over $N_{config}$ configurations, the variance is
\begin{align}
  \label{eqn:err1}
  \sigma ^2\qty[\ct]= \frac{1}{N_{config}}\qty[\ev{x_{l}^2x_{k}^2}-\ev{x_{l}x_{k}}^2],
\end{align}
where $\ev{x_{l}x_{k}}$ is precisely $\ct$, while $\ev{x_{l}^2x_{k}^2}$ is the two points correlation function of the
$\x ^2$ operator. \\Like we did for the correlator of $\hat{x}$, we can write $\ev{x_{l}^{2}x_{k}^{2}}$ using
the operator formalism and, following the same steps, recover its dominant contribution for large $N$
\begin{align}
  \ev{x_{l}^{2}x_{k}^{2}} = &\frac{1}{\Tr[\T{N}]}\Tr[\T{N-\lk}\x^{2}\T{\lk}\x^{2}] \notag\\
  &\approx \abs{\mel{\Et{0}}{\x ^2}{\Et{0}}}^2\label{eqn:xlxkq}.
\end{align}
Before we continue discussing the exponential problem, it is worth noting that for large values of $N$, as the distance $|l-k|$ grows, the variance (equation \ref{eqn:err1})
will be dominated by the constant value of $\ev{x_{l}^{2}x_{k}^{2}}$\footnote{Equation \ref{eqn:correlators} shows that $\ct$ is suppressed exponentially with $N$ when $|l-k|$ is large enough.}
\begin{align}
  \label{eqn:varmatrix}
  \sigma^{2}[\ct]\approx \frac{ \abs{\mel{\Et{0}}{\x ^2}{\Et{0}}}^2}{N_{config}},
\end{align}
meaning that it would be possible to use the variance of our numerical computation to extract the ground state expectation value of $\x^{2}$.\\
On the other hand, if we focus on the region where $N$ and $\lk$ are large but $\lk \ll \nicefrac{N}{2}$, equation \ref{eqn:correlators} implies that the correlators decrease very rapidly with $\lk$
\footnote{$\cosh\qty(x)\sim \nicefrac{e^{x}}{2}$ as $x\to\infty$.}
\begin{align}
  \label{eqn:errct}
  \ct \approx \matb e ^{-a\lk \DEt}.
\end{align}
The noise to signal ratio, \textit{i.e.} the relative error, therefore grows exponentially with the distance of the sources
\begin{align}
  \label{eqn:relerr}
 \frac{\sigma ^2\qty[\ct]}{\ct ^2}\approx \frac{1}{N_{config}}\frac{\abs{\mel{\Et{0}}{\x ^2}{\Et{0}}}^2}{\abs{\mel{\Et{0}}{\x}{\Et{1}}}^4}e ^{2a\lk\DEt}.
\end{align}

This behaviour greatly limits the number of correlators we can actually use to calculate the energy gaps and matrix elements (equations \ref{eqn:energy} and \ref{eqn:matelem}) for a given number of configurations, as the value of the observables is eventually going to be dominated by their error.
A possible solution would be to exponentially increase the number of configurations, which would heavily worsen the computational cost.

More generally, the origin of this problem in QFTs is to be found in the quantization of the energy spectrum which leads to the presence of a mass gap in the theory.

