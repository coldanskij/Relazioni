\section{Introduction}%
\label{sec:intro}

The classical QCD action, in the massless fermions limit, has an accidental global symmetry
called chiral simmetry. The $U(1)_{L}\cp U(1)_{R}$ subgroup is broken to $U(1)_{B}$\footnote{$L$ and $R$ stand for left and right, $B$ stands for baryonic.} as a consequence of its quantization via the path integral.
This is called the \textit{chiral anomaly}: it happens because the integration measure is not invariant
under the action of the axial group $U(1)_{A}$ and, ultimately, it is the reason for the unexpectedly large mass of the
$\eta'$ meson $m_{\eta'}=\SI{0.958}{GeV}$, which is $0$ in the classical theory.

The Witten-Veneziano formula aims at providing an explaination for the mass of the
$\eta' $ meson by connecting it to non trivial topological fluctuations of the guage fields, encoded by the topological susceptibility $\chi$ in pure Yang-Mills theories. The formula takes its simplest
form in the chiral limit
\begin{align}
  \label{eqn:chiralm}m_{\eta' }^{chiral} = \sqrt{\frac{2N_{f}\chi}{F_{\eta'}^2}}
\end{align}
where $N_{f}$ is the flavor number and $F_{\eta'}$ is the leptonic decay constant of the $\eta'$.
The chiral contribution proves to be the dominant one even in the non chiral limit
\begin{align}
  \label{eqn:nonchiralmass}m_{\eta'}= \sqrt{\frac{2N_{f}\chi}{F_{\eta'}^2} +2m_{K}^2-m_{\eta}^2}
\end{align}
where $m_{K}$ and $m_{\eta}$ are the masses of the $K ^{0}$ and $\eta$ mesons.

The nature of topological susceptibility lies in the fact that in $SU$ invariant theories the allowed
field configurations form disjoint clusters characterized by the topological charge $Q$, which takes integer values and is
non zero on average\footnote{This is the reason behind the integration measure not being $U(1)_{A}$ invariant.}.
The topological susceptibility $\chi$ is the variance of $Q$.

Moreover, since $\chi = 0$ both in the classical theory and in perturbative quantization,
measurements of $m_{\eta'}$ are a strong confirmation of non perturbative QCD quantization.

This work is heavily based on \cite{Ce}, both for the theoretical formulation and for the computational implementation.
We will consider Wilson discretization to define lattice QCD and, through the Wilson flow definition of the topological charge,
we will compute the leading contribution to the topological susceptibility as the continuum limit of the values obtained
on three different lattices; then we will use it to compute the $\eta'$ mass.
